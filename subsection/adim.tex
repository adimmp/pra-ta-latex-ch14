\subsection{Pengertian Plagiarisme}
Plagiat adalah perbuatan sengaja atau tidak sengaja dalam memperoleh atau mencoba memperoleh kredit atau nilai untuk suatu karya ilmiah, dengan mengutip sebagian atau seluruh karya dan atau karya ilmiah pihak lain yang diakui sebagai karya ilmiahnya, tanpa menyatakan sumber secara tepat dan memadai (Peraturan Menteri Pendidikan RI Nomor 17 Tahun 2010).

\subsection{Ruang Lingkup Plagiarisme}
\begin{enumerate}
  \item \textbf{Mengutip kata-kata atau kalimat orang lain tanpa menggunakan tanda kutip dan tanpa menyebutkan identitas sumbernya.}
  \item \textbf{Menggunakan gagasan, pandangan atau teori orang lain tanpa menyebutkan identitas sumbernya.}
  \item \textbf{Menggunakan fakta (data, informasi) milik orang lain tanpa menyebutkan identitas sumbernya.}
  \item \textbf{Mengakui tulisan orang lain sebagai tulisan sendiri.
  \item Melakukan parafrase (mengubah kalimat orang lain ke dalam susunan kalimat sendiri tanpa mengubah idenya) tanpa menyebutkan identitas sumbernya.}
  \item \textbf{Menyerahkan suatu karya ilmiah yang dihasilkan dan /atau telah dipublikasikan oleh pihak lain seolah-olah sebagai karya sendiri.}
\end{enumerate}

\subsection{Tips Menghindari Plagiarisme}
\begin{enumerate}
    \item \textbf{Kutipan} 
     \newline Menyertakan darimana sumber teori atau gagasan itu berasal dalam proses penulisan yang sedang dilakukan seseorang. Ada beberapa jenis bentuk kutipan yang umum digunakan yaitu Harvard citation style, Chicago style, The Modern Language style, dan APA style.
     \item \textbf{Parafrase}
     \newline Teknik penulisan yang menggunakan gagasan orang lain dengan mengungkapkannya dengan
kata sendiri.


\end{enumerate}