\section{Etika dan Plagiarisme}

\subsection{Pengertian Ethics}
\begin{itemize}
    \item Konsep nilai yang mengarah pada perilaku yang baik dan pantas.
    \item Terkait dengan norma, moralitas, pranata, baik kemanusiaan maupun agama.
\end{itemize}

\subsection{Apa yang perlu ditulis}
Yang perlu ditulis jika mengambil referensi dari paper atau jurnal milik penulis lain ialah :
\begin{itemize}
  \item Nama Penulis
   \newline Nama penulis dari paper atau jurnal yang ingin dicantumkan dapat ditulis dengan menulis nama belakang terlebih dahulu, lalu diikuti tanda ‘ , ‘ dilanjutkan dengan nama depan.
  \item Judul
  \newline Nama penulis dari paper atau jurnal yang ingin dicantumkan dapat ditulis dengan menulis nama belakang terlebih dahulu, lalu diikuti tanda ‘ , ‘ dilanjutkan dengan nama depan.
  \item Tahun
  \newline Tahun penulisan karya yang dijadikan referensi dituliskan di dalam tanda baca kurung ‘ ( . . . . ) ‘.
  \item Halaman
  \newline Halaman yang dicantumkan adalah halaman mana yang diambil sebagai referensi
\end{itemize}

\subsection{Hal yang perlu diperhatikan}
Kode Etik Penulis:
\begin{itemize}
    \item Melahirkan karya orisinal, bukan jiplakan
    \item Data yang ditampilkan harus asli, atau disitasi dengan baik dari penerbitan sebelumnya
    \item Menjaga kebenaran dan manfaat serta makna informasi yang disebarkan sehingga tidak menyesatkan
    \item Menulis secara cermat, teliti, dan tepat
    \item Bertanggung jawab secara akademis atas tulisannya
    \item Dalam kaitan dengan berkala ilmiah, menjadi kewajiban bagi penulis untuk mengikuti gaya selingkung yang ditetapkan berkala yang dituju
    \item Menerima saran-saran perbaikan dari editor berkala yang dituju
    \item Menjunjung tinggi hak, pendapat atau temuan orang lain
    \item Menyadari sepenuhnya untuk tidak melakukan pelanggaran ilmiah (Falsifikasi; Fabrikasi; Plagiat)
\end{itemize}
Cite Sources Appropriately :
\begin{itemize}
    \item Kutipan langsung: Tempatkan teks kata demi kata dari sumber lain dalam tanda kutip. Indentasi teks untuk kutipan yang lebih panjang. Sertakan kutipan ke sumber aslinya.
    \item Parafrase atau ringkasan: Sertakan kutipan ketika menyatakan kembali atau meringkas informasi dari sumber lain, termasuk ide, proses, argumen, atau kesimpulan.
    \item Data, hasil penelitian, informasi, grafik, atau tabel: Mengutip sumber asli ketika merujuk, mengadaptasi, atau menggunakan kembali informasi apa pun dari sumber lain
\end{itemize}

\subsection{Pengertian Plagiarisme}
Plagiat adalah perbuatan sengaja atau tidak sengaja dalam memperoleh atau mencoba memperoleh kredit atau nilai untuk suatu karya ilmiah, dengan mengutip sebagian atau seluruh karya dan atau karya ilmiah pihak lain yang diakui sebagai karya ilmiahnya, tanpa menyatakan sumber secara tepat dan memadai (Peraturan Menteri Pendidikan RI Nomor 17 Tahun 2010).

\subsection{Ruang Lingkup Plagiarisme}
\begin{enumerate}
  \item \textbf{Mengutip kata-kata atau kalimat orang lain tanpa menggunakan tanda kutip dan tanpa menyebutkan identitas sumbernya.}
  \item \textbf{Menggunakan gagasan, pandangan atau teori orang lain tanpa menyebutkan identitas sumbernya.}
  \item \textbf{Menggunakan fakta (data, informasi) milik orang lain tanpa menyebutkan identitas sumbernya.}
  \item \textbf{Mengakui tulisan orang lain sebagai tulisan sendiri.
  \item Melakukan parafrase (mengubah kalimat orang lain ke dalam susunan kalimat sendiri tanpa mengubah idenya) tanpa menyebutkan identitas sumbernya.}
  \item \textbf{Menyerahkan suatu karya ilmiah yang dihasilkan dan /atau telah dipublikasikan oleh pihak lain seolah-olah sebagai karya sendiri.}
\end{enumerate}

\subsection{Tips Menghindari Plagiarisme}
\begin{enumerate}
    \item \textbf{Kutipan} 
     \newline Menyertakan darimana sumber teori atau gagasan itu berasal dalam proses penulisan yang sedang dilakukan seseorang. Ada beberapa jenis bentuk kutipan yang umum digunakan yaitu Harvard citation style, Chicago style, The Modern Language style, dan APA style.
     \item \textbf{Parafrase}
     \newline Teknik penulisan yang menggunakan gagasan orang lain dengan mengungkapkannya dengan
kata sendiri.
\end{enumerate}

\subsection{Pencegahan Plagiarisme}
\begin{enumerate}
    \item \textbf{Karya mahasiswa (skripsi, tesis dan disertasi) dilampiri dengan surat pernyataan dari
    yang bersangkutan, yang menyatakan bahwa karya ilmiah tersebut tidak mengandung
    unsur plagiat.}
    \item \textbf{Pimpinan Perguruan Tinggi berkewajiban mengunggah semua karya ilmiah yang dihasilkan
    di lingkungan perguruan tingginya, seperti portal Garuda atau portal lain
    yang ditetapkan oleh Direktorat Pendidikan Tinggi.} 
    \item \textbf{Sosialisasi terkait dengan UU Hak Cipta No. 19 Tahun 2002 dan Permendiknas No. 17
    Tahun 2010 kepada seluruh masyarakat akademis.} 
\end{enumerate}

\subsection{Contoh Penulisan}
Berikut merupakan contoh penulisan dengan format APA (American Psychological Association) :
\begin{itemize}
    \item \textbf{Penulisan Daftar Pustaka Jurnal Bentuk Cetak (Satu Penulis)}
    \begin{itemize}
        \item Ready, R. (2000). Mothers’ personality and its interaction with child temperament as predictors of parenting behavior. Journal of Personality and Social Psychology, 79, 274-285.
        \item Jacoby, W. G. (1994). Public attitudes toward government spending. American Journal of Political Science, 38(2), 336-361.
        \item  Dubeck, L. (1990). Science ficPon aids science teaching. Physics Teacher, 28, 316-318.
    \end{itemize}
    \item \textbf{Penulisan Daftar Pustaka Jurnal Bentuk Cetak (2 atau Lebih Pengarang)}
    \begin{itemize}
        \item Yonkers, K. A., Ramin, S. M., Rush, A. J., Navarrete, C. A., Carmody, T., March, D., Leveno, K. J. (2001). Onset and persistence of postpartum depression in an inner-city maternal health clinic system. American Journal of Psychiatry, 158(11), 1856-1863. doi:10.1176/appi.ajp.158.11.1856        
    \end{itemize}
    \item \textbf{Penulisan Daftar Pustaka Jurnal Online atau Internet}
    \begin{itemize}
        \item Spreer, P., Rauschnabel, P.A. (2016, September). Selling with technology: Understanding the resistance to mobile sales assistant use in retailing. Journal of Personal Selling \& Sales Management, 36(3), 240-263. doi:10.1080/08853134.2016.1208100        
    \end{itemize}
    \item \textbf{Cara Penulisan Daftar Pustaka dari Jurnal Internet (tanpa doi)}
    \begin{itemize}
        \item Jameson, J. (2013). E-Leadership in higher education: The fifth “age” of educational technology research. British Journal of Educational Technology, 44(6), 889-915. Retrieved from http://onlinelibrary.wiley.com/journal/10.1111/(ISSN)14678535       
    \end{itemize}
\end{itemize}



Sedangkan untuk yang dibawah ini merupakan contoh penulisan dengan format MLA (Modern Language Association ) :

\begin{itemize}
    \item \textbf{Penulisan Daftar Pustaka Jurnal Bentuk Cetak (Satu Penulis)}
    \begin{itemize}
        \item Smith, John. “Studies in pop rocks and Coke.” Weird Science 12 (2009): 78-93. Print.
    \end{itemize}
    \item \textbf{Penulisan Daftar Pustaka Jurnal Bentuk Cetak (2 atau Lebih Pengarang)}
    \begin{itemize}
        \item Smith, John, et al. “Studies in pop rocks and Coke.” Weird Science 12 (2009): 78-93. Print.        
    \end{itemize}
    \item \textbf{Cara Penulisan Daftar Pustaka dari Jurnal Internet}
    \begin{itemize}
        \item Johansson, Sara. “A Participle Account of Blackfoot Relative Clauses.” The Canadian Journal of Linguistics 58.2 (2013): 217-38. Project Muse. Web. 5 Mar. 2015.
    \end{itemize}
\end{itemize}


\subsection{Referensi Penulisan}

\begin{enumerate}
    \item https://journals.ieeeauthorcenter.ieee.org/become-an-ieee-journal-author/publishing-ethics/
    \item http://staffnew.uny.ac.id/upload/197902072014041001/pengabdian/KODEETIKPENULISANILMIAH.pdf
    \item https://salamadian.com/cara-penulisan-daftar-pustaka-dari-jurnal/
\end{enumerate}