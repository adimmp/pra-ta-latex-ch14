\documentclass{article}
\usepackage{indentfirst}
\title{Title of my document}
\date{2013-09-01}
\author{John Doe}

\begin{document}

\maketitle
\pagenumbering{gobble}
\newpage
\pagenumbering{arabic}

\section{Etika dan Plagiarisme}

\subsection{Pengertian Plagiarisme}

Plagiat adalah perbuatan sengaja atau tidak sengaja dalam memperoleh atau mencoba memperoleh kredit atau nilai untuk suatu karya ilmiah, dengan mengutip sebagian atau seluruh karya dan atau karya ilmiah pihak lain yang diakui sebagai karya ilmiahnya, tanpa menyatakan sumber secara tepat dan memadai (Peraturan Menteri Pendidikan RI Nomor 17 Tahun 2010).

\subsection{Ruang Lingkup Plagiarisme}
\begin{enumerate}
  \item Mengutip kata-kata atau kalimat orang lain tanpa menggunakan tanda kutip dan tanpa menyebutkan identitas sumbernya.
  \item Menggunakan gagasan, pandangan atau teori orang lain tanpa menyebutkan identitas sumbernya.
  \item Menggunakan fakta (data, informasi) milik orang lain tanpa menyebutkan identitas sumbernya.
  \item Mengakui tulisan orang lain sebagai tulisan sendiri.
  \item Melakukan parafrase (mengubah kalimat orang lain ke dalam susunan kalimat sendiri tanpa mengubah idenya) tanpa menyebutkan identitas sumbernya.
  \item Menyerahkan suatu karya ilmiah yang dihasilkan dan /atau telah dipublikasikan oleh pihak lain seolah-olah sebagai karya sendiri.
\end{enumerate}

\subsection{Ruang Lingkup Plagiarisme}
\begin{enumerate}
    \item 
\end{enumerate}

\end{document}