\documentclass{article}
\setlength{\parindent}{4em}
\title{Title of my document}
\date{2013-09-01}
\author{John Doe}

\begin{document}

\maketitle
\pagenumbering{gobble}
\newpage
\pagenumbering{arabic}

\section{Etika dan Plagiarisme}

\subsection{Pengertian Ethics}
\begin{itemize}
    \item Konsep nilai yang mengarah pada perilaku yang baik dan pantas.
    \item Terkait dengan norma, moralitas, pranata, baik kemanusiaan maupun agama.
\end{itemize}

\subsection{Apa yang perlu ditulis}
Yang perlu ditulis jika mengambil referensi dari paper atau jurnal milik penulis lain ialah :
\begin{itemize}
  \item Nama Penulis
   \newline Nama penulis dari paper atau jurnal yang ingin dicantumkan dapat ditulis dengan menulis nama belakang terlebih dahulu, lalu diikuti tanda ‘ , ‘ dilanjutkan dengan nama depan.
  \item Judul
  \newline Nama penulis dari paper atau jurnal yang ingin dicantumkan dapat ditulis dengan menulis nama belakang terlebih dahulu, lalu diikuti tanda ‘ , ‘ dilanjutkan dengan nama depan.
  \item Tahun
  \item Halaman
\end{itemize}

\subsection{Hal yang perlu diperhatikan}
Kode Etik Penulis:
\begin{itemize}
    \item Melahirkan karya orisinal, bukan jiplakan
    \item Data yang ditampilkan harus asli, atau disitasi dengan baik dari penerbitan sebelumnya
    \item Menjaga kebenaran dan manfaat serta makna informasi yang disebarkan sehingga tidak menyesatkan
    \item Menulis secara cermat, teliti, dan tepat
    \item Bertanggung jawab secara akademis atas tulisannya
    \item Dalam kaitan dengan berkala ilmiah, menjadi kewajiban bagi penulis untuk mengikuti gaya selingkung yang ditetapkan berkala yang dituju
    \item Menerima saran-saran perbaikan dari editor berkala yang dituju
    \item Menjunjung tinggi hak, pendapat atau temuan orang lain
    \item Menyadari sepenuhnya untuk tidak melakukan pelanggaran ilmiah (Falsifikasi; Fabrikasi; Plagiat)
\end{itemize}
Cite Sources Appropriately :
\begin{itemize}
    \item Kutipan langsung: Tempatkan teks kata demi kata dari sumber lain dalam tanda kutip. Indentasi teks untuk kutipan yang lebih panjang. Sertakan kutipan ke sumber aslinya.
    \item Parafrase atau ringkasan: Sertakan kutipan ketika menyatakan kembali atau meringkas informasi dari sumber lain, termasuk ide, proses, argumen, atau kesimpulan.
    \item Data, hasil penelitian, informasi, grafik, atau tabel: Mengutip sumber asli ketika merujuk, mengadaptasi, atau menggunakan kembali informasi apa pun dari sumber lain
\end{itemize}

\end{document}