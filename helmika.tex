\documentclass{article}
\usepackage[utf8]{inputenc}

\title{testing123}
\author{helmika.zero }
\date{November 2021}

\begin{document}

\maketitle

\pagenumbering{gobble}
\newpage
\pagenumbering{arabic}


\section{Etika dan Plagiarisme}

\subsection{Pencegahan Plagiarisme}
\begin{enumerate}
    \item \textbf{Karya mahasiswa (skripsi, tesis dan disertasi) dilampiri dengan surat pernyataan dari
    yang bersangkutan, yang menyatakan bahwa karya ilmiah tersebut tidak mengandung
    unsur plagiat.}
    \item \textbf{Pimpinan Perguruan Tinggi berkewajiban mengunggah semua karya ilmiah yang dihasilkan
    di lingkungan perguruan tingginya, seperti portal Garuda atau portal lain
    yang ditetapkan oleh Direktorat Pendidikan Tinggi.} 
    \item \textbf{Sosialisasi terkait dengan UU Hak Cipta No. 19 Tahun 2002 dan Permendiknas No. 17
    Tahun 2010 kepada seluruh masyarakat akademis.} 
\end{enumerate}

\subsection{Pencegahan Plagiarisme}
Berikut merupakan contoh penulisan dengan format APA (American Psychological Association) :
\begin{itemize}
    \item \textbf{Penulisan Daftar Pustaka Jurnal Bentuk Cetak (Satu Penulis)}
    \begin{itemize}
        \item Ready, R. (2000). Mothers’ personality and its interaction with child temperament as predictors of parenting behavior. Journal of Personality and Social Psychology, 79, 274-285.
        \item Jacoby, W. G. (1994). Public attitudes toward government spending. American Journal of Political Science, 38(2), 336-361.
        \item  Dubeck, L. (1990). Science ficPon aids science teaching. Physics Teacher, 28, 316-318.
    \end{itemize}
    \item \textbf{nulisan Daftar Pustaka Jurnal Bentuk Cetak (2 atau Lebih Pengarang)}
    \begin{itemize}
        \item Yonkers, K. A., Ramin, S. M., Rush, A. J., Navarrete, C. A., Carmody, T., March, D., Leveno, K. J. (2001). Onset and persistence of postpartum depression in an inner-city maternal health clinic system. American Journal of Psychiatry, 158(11), 1856-1863. doi:10.1176/appi.ajp.158.11.1856        
    \end{itemize}
    \item \textbf{Penulisan Daftar Pustaka Jurnal Online atau Internet}
    \begin{itemize}
        \item Spreer, P., Rauschnabel, P.A. (2016, September). Selling with technology: Understanding the resistance to mobile sales assistant use in retailing. Journal of Personal Selling \& Sales Management, 36(3), 240-263. doi:10.1080/08853134.2016.1208100        
    \end{itemize}
    \item \textbf{Cara Penulisan Daftar Pustaka dari Jurnal Internet (tanpa doi)}
    \begin{itemize}
        \item Jameson, J. (2013). E-Leadership in higher education: The fifth “age” of educational technology research. British Journal of Educational Technology, 44(6), 889-915. Retrieved from http://onlinelibrary.wiley.com/journal/10.1111/(ISSN)14678535       
    \end{itemize}
\end{itemize}



Sedangkan untuk yang dibawah ini merupakan contoh penulisan dengan format MLA (Modern Language Association ) :

\begin{itemize}
    \item \textbf{Penulisan Daftar Pustaka Jurnal Bentuk Cetak (Satu Penulis)}
    \begin{itemize}
        \item Smith, John. “Studies in pop rocks and Coke.” Weird Science 12 (2009): 78-93. Print.
    \end{itemize}
    \item \textbf{Penulisan Daftar Pustaka Jurnal Bentuk Cetak (2 atau Lebih Pengarang)}
    \begin{itemize}
        \item Smith, John, et al. “Studies in pop rocks and Coke.” Weird Science 12 (2009): 78-93. Print.        
    \end{itemize}
    \item \textbf{Cara Penulisan Daftar Pustaka dari Jurnal Internet}
    \begin{itemize}
        \item Johansson, Sara. “A Participle Account of Blackfoot Relative Clauses.” The Canadian Journal of Linguistics 58.2 (2013): 217-38. Project Muse. Web. 5 Mar. 2015.
    \end{itemize}
\end{itemize}


\subsection{Referensi Penulisan}

\begin{enumerate}
    \item https://journals.ieeeauthorcenter.ieee.org/become-an-ieee-journal-author/publishing-ethics/
    \item http://staffnew.uny.ac.id/upload/197902072014041001/pengabdian/KODEETIKPENULISANILMIAH.pdf
    \item https://salamadian.com/cara-penulisan-daftar-pustaka-dari-jurnal/
\end{enumerate}

\end{document}
